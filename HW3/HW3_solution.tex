%%%%%%%%%%%%%%%%% DO NOT CHANGE HERE %%%%%%%%%%%%%%%%%%%% {
    \documentclass[12pt,letterpaper]{article}
    \usepackage{fullpage}
    \usepackage[top=2cm, bottom=4.5cm, left=2.5cm, right=2.5cm]{geometry}
    \usepackage{amsmath,amsthm,amsfonts,amssymb,amscd}
    \usepackage{lastpage}
    \usepackage{enumerate}
    \usepackage{fancyhdr}
    \usepackage{mathrsfs}
    \usepackage{xcolor}
    \usepackage{graphicx}
    \usepackage{listings}
    \usepackage{hyperref}
    \usepackage{float} 
    \usepackage{subfigure}
    \definecolor{codegreen}{rgb}{0,0.6,0}
    \definecolor{codegray}{rgb}{0.5,0.5,0.5}
    \definecolor{codepurple}{rgb}{0.58,0,0.82}
    \definecolor{backcolour}{rgb}{0.95,0.95,0.92}
    
    \lstdefinestyle{mystyle}{
        backgroundcolor=\color{backcolour},   
        commentstyle=\color{codegreen},
        keywordstyle=\color{magenta},
        numberstyle=\tiny\color{codegray},
        stringstyle=\color{codepurple},
        basicstyle=\ttfamily\footnotesize,
        breakatwhitespace=false,         
        breaklines=true,                 
        captionpos=b,                    
        keepspaces=true,                 
        numbers=left,                    
        numbersep=5pt,                  
        showspaces=false,                
        showstringspaces=false,
        showtabs=false,                  
        tabsize=2
    }

    \hypersetup{%
      colorlinks=true,
      linkcolor=blue,
      linkbordercolor={0 0 1}
    }
    
    \setlength{\parindent}{0.0in}
    \setlength{\parskip}{0.05in}
    %%%%%%%%%%%%%%%%%%%%%%%%%%%%%%%%%%%%%%%%%%%%%%%%%%%%%%%%%% }
    
    %%%%%%%%%%%%%%%%%%%%%%%% CHANGE HERE %%%%%%%%%%%%%%%%%%%% {
    \newcommand\course{ECE 271A}
    \newcommand\semester{Fall 2019}
    \newcommand\hwnumber{\#3}                 % <-- ASSIGNMENT #
    \newcommand\NetIDa{Jiaming Lai}           % <-- YOUR NAME
    \newcommand\NetIDb{A53314574}           % <-- STUDENT ID #
    %%%%%%%%%%%%%%%%%%%%%%%%%%%%%%%%%%%%%%%%%%%%%%%%%%%%%%%%%% }
    
    %%%%%%%%%%%%%%%%% DO NOT CHANGE HERE %%%%%%%%%%%%%%%%%%%% {
    \pagestyle{fancyplain}
    \headheight 35pt
    \lhead{\NetIDa}
    \lhead{\NetIDa\\\NetIDb}                 
    \chead{\textbf{\Large Assignment \hwnumber}}
    \rhead{\course \\ \semester}
    \lfoot{}
    \cfoot{}
    \rfoot{\small\thepage}
    \headsep 1.5em
    %%%%%%%%%%%%%%%%%%%%%%%%%%%%%%%%%%%%%%%%%%%%%%%%%%%%%%%%%% }
    
    \begin{document}
     
    \section*{Computer Problem Solution}
    % If the Problem is divided into items, use "enumerate"
    \begin{enumerate}[a)]
        %%%%%%%%%%%%%%%%%%%%%%%%%%%%%%%%%%%%%%%%%%%%%%%%%%%%%%%%%%
        % Problem (a)
        %%%%%%%%%%%%%%%%%%%%%%%%%%%%%%%%%%%%%%%%%%%%%%%%%%%%%%%%%%
        \item 
        sdf
        
    \end{enumerate}

    \section*{Appendix}
    The following is the Matlab code.

    \subsection{HW3\_solution.m}
    \lstset{style=mystyle}
    \lstinputlisting[language=Matlab]{HW3_solution.m}

    \subsection{fun\_general.m}
    This function receives parameters: train data, array of alpha and strategy we want to choose,
    and compute error using MAP-BDR, Bayes-BDR and ML-BDR.
    \lstset{style=mystyle}
    \lstinputlisting[language=Octave]{myfunction/fun_general.m}

    \subsection{fun\_bayesBDR.m}
    This function performs Bayes-BDR.
    \lstset{style=mystyle}
    \lstinputlisting[language=Octave]{myfunction/fun_bayesBDR.m}

    \subsection{fun\_mapBDR.m}
    This function performs MAP-BDR.
    \lstset{style=mystyle}
    \lstinputlisting[language=Octave]{myfunction/fun_mapBDR.m}

    \subsection{fun\_mlBDR.m}
    This function performs ML-BDR.
    \lstset{style=mystyle}
    \lstinputlisting[language=Octave]{myfunction/fun_mlBDR.m}

    \subsection{fun\_zigzag.m}
    This function compute DCT coefficients and save it as DCT\_coefficience.mat.
    \lstset{style=mystyle}
    \lstinputlisting[language=Octave]{myfunction/fun_zigzag.m}

    \subsection{fun\_mvgaussian.m}
    This function perfroms multi-gaussian.
    \lstset{style=mystyle}
    \lstinputlisting[language=Octave]{myfunction/fun_mvgaussian.m}

    \subsection{fun\_cov.m}
    \lstset{style=mystyle}
    \lstinputlisting[language=Octave]{myfunction/fun_cov.m}

    \subsection{fun\_mean.m}
    \lstset{style=mystyle}
    \lstinputlisting[language=Octave]{myfunction/fun_mean.m}

    \end{document}