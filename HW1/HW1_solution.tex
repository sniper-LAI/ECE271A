%%%%%%%%%%%%%%%%% DO NOT CHANGE HERE %%%%%%%%%%%%%%%%%%%% {
    \documentclass[12pt,letterpaper]{article}
    \usepackage{fullpage}
    \usepackage[top=2cm, bottom=4.5cm, left=2.5cm, right=2.5cm]{geometry}
    \usepackage{amsmath,amsthm,amsfonts,amssymb,amscd}
    \usepackage{lastpage}
    \usepackage{enumerate}
    \usepackage{fancyhdr}
    \usepackage{mathrsfs}
    \usepackage{xcolor}
    \usepackage{graphicx}
    \usepackage{listings}
    \usepackage{hyperref}
    
    \hypersetup{%
      colorlinks=true,
      linkcolor=blue,
      linkbordercolor={0 0 1}
    }
    
    \setlength{\parindent}{0.0in}
    \setlength{\parskip}{0.05in}
    %%%%%%%%%%%%%%%%%%%%%%%%%%%%%%%%%%%%%%%%%%%%%%%%%%%%%%%%%% }
    
    %%%%%%%%%%%%%%%%%%%%%%%% CHANGE HERE %%%%%%%%%%%%%%%%%%%% {
    \newcommand\course{ECE 271A}
    \newcommand\semester{Fall 2019}
    \newcommand\hwnumber{\#1}                 % <-- ASSIGNMENT #
    \newcommand\NetIDa{Jiaming Lai}           % <-- YOUR NAME
    \newcommand\NetIDb{A53314574}           % <-- STUDENT ID #
    %%%%%%%%%%%%%%%%%%%%%%%%%%%%%%%%%%%%%%%%%%%%%%%%%%%%%%%%%% }
    
    %%%%%%%%%%%%%%%%% DO NOT CHANGE HERE %%%%%%%%%%%%%%%%%%%% {
    \pagestyle{fancyplain}
    \headheight 35pt
    \lhead{\NetIDa}
    \lhead{\NetIDa\\\NetIDb}                 
    \chead{\textbf{\Large Assignment \hwnumber}}
    \rhead{\course \\ \semester}
    \lfoot{}
    \cfoot{}
    \rfoot{\small\thepage}
    \headsep 1.5em
    %%%%%%%%%%%%%%%%%%%%%%%%%%%%%%%%%%%%%%%%%%%%%%%%%%%%%%%%%% }
    
    \begin{document}
     
    \section*{Computer Problem Solution}
    % If the Problem is divided into items, use "enumerate"
    \begin{enumerate}[a)]
        \item 
        Using the training data in \textbf{TrainingSamplesDCT8.mat}, what are reasonable estimates for the prior probabilities?
        
        \textbf{Solution}:\\
        Two priors probabilities, $P_Y(cheetah)$ and $P_Y(grass)$, could be estimated based on the number of vectors in the training set.
        The estimation of $P_Y(cheetah)$ and $P_Y(grass)$ are: 
        \begin{equation}
            P_Y(cheetah) = N_{FG} / (N_{FG} + N_{BG})
        \end{equation}
        \begin{equation}
            P_Y(grass) = N_{BG} / (N_{FG} + N_{BG})
        \end{equation}
        where 
        \begin{itemize}
            \item[] $N_{BG}$ is the number of vectors in matrix \textbf{TrainsampleDCT\_BG}
            \item[] $N_{FG}$ is the number of vectors in matrix \textbf{TrainsampleDCT\_FG}
        \end{itemize}
        
        \item 
        Using the training data in \textbf{TrainingSamplesDCT8.mat}, compute and plot the index histograms
        $P_{X|Y}(x|cheetah)$ and $P_{X|Y}(x|grass)$. 
        
        \textbf{Solution}:\\
        The index histograms is the following picture:
        \begin{figure}[h]
            \centering
            \includegraphics[scale=0.28]{Images/histograms.jpg}
            \caption{index histograms}
        \end{figure}\\
        The value of $P_{X|Y}(x|cheetah)$ and $P_{X|Y}(x|grass)$ is showed as following histograms:
        \begin{figure}[h]
            \centering
            \includegraphics[scale=0.28]{Images/histograms.jpg}
            \caption{index histograms}
        \end{figure}
        \begin{enumerate}[1.]
            \item 
            \begin{equation}
                \forall n(P(n) \rightarrow Q(n)),
            \end{equation}
           where
           \begin{itemize}
                \item[] $P(n)$ is ``$n$ is an odd integer'' and
                \item[] $Q(n)$ is ``$n^2$ is odd.''
           \end{itemize}
            
            \item 
            Assume $P(n)$  is true.
            
            \item 
            By definition, an odd integer is $n = 2k + 1$, 
            where $k$ is some integer.
    
            \item
            \begin{align*}
                n^2 &= (2k + 1)^2 \\
                    &= 4k^2 + 4k + 1 \\
                    &=  2(2k^2 + 2k) + 1
            \end{align*}
            
            \item 
            $\therefore n^2$ is an odd integer. $\qed$
        \end{enumerate}
        
        
    
        \item Let $A = \{1,2,3\}$ and $B = \{1,2,3,\{1,2,3\}\}$:
            
        Then, $A \in B$ and $A \subseteq B$.
        
        \item Let $A = \{1, 3, 5\}$, $B = \{1,2,3,\}$, and universe $U = \{1,2,3,4,5\}$:
        \begin{align*}
            A \cup B    &= \{1,2,3,5\}, \\
            A \cap B    &= \{1,3\}, \\
            A - B       &= \{5\},\\
            \bar{A}     &= \{2,4\},\\
            A - A       &= \emptyset .
        \end{align*}
    
    \end{enumerate}
    
    \end{document}