%%%%%%%%%%%%%%%%% DO NOT CHANGE HERE %%%%%%%%%%%%%%%%%%%% {
    \documentclass[12pt,letterpaper]{article}
    \usepackage{fullpage}
    \usepackage[top=2cm, bottom=4.5cm, left=2.5cm, right=2.5cm]{geometry}
    \usepackage{amsmath,amsthm,amsfonts,amssymb,amscd}
    \usepackage{lastpage}
    \usepackage{enumerate}
    \usepackage{fancyhdr}
    \usepackage{mathrsfs}
    \usepackage{xcolor}
    \usepackage{graphicx}
    \usepackage{listings}
    \usepackage{hyperref}
    \usepackage{float} 
    \usepackage{subfigure}
    \definecolor{codegreen}{rgb}{0,0.6,0}
    \definecolor{codegray}{rgb}{0.5,0.5,0.5}
    \definecolor{codepurple}{rgb}{0.58,0,0.82}
    \definecolor{backcolour}{rgb}{0.95,0.95,0.92}
    
    \lstdefinestyle{mystyle}{
        backgroundcolor=\color{backcolour},   
        commentstyle=\color{codegreen},
        keywordstyle=\color{magenta},
        numberstyle=\tiny\color{codegray},
        stringstyle=\color{codepurple},
        basicstyle=\ttfamily\footnotesize,
        breakatwhitespace=false,         
        breaklines=true,                 
        captionpos=b,                    
        keepspaces=true,                 
        numbers=left,                    
        numbersep=5pt,                  
        showspaces=false,                
        showstringspaces=false,
        showtabs=false,                  
        tabsize=2
    }

    \hypersetup{%
      colorlinks=true,
      linkcolor=blue,
      linkbordercolor={0 0 1}
    }
    
    \setlength{\parindent}{0.0in}
    \setlength{\parskip}{0.05in}
    %%%%%%%%%%%%%%%%%%%%%%%%%%%%%%%%%%%%%%%%%%%%%%%%%%%%%%%%%% }
    
    %%%%%%%%%%%%%%%%%%%%%%%% CHANGE HERE %%%%%%%%%%%%%%%%%%%% {
    \newcommand\course{ECE 271A}
    \newcommand\semester{Fall 2019}
    \newcommand\hwnumber{\#5}                 % <-- ASSIGNMENT #
    \newcommand\NetIDa{Jiaming Lai}           % <-- YOUR NAME
    \newcommand\NetIDb{A53314574}           % <-- STUDENT ID #
    %%%%%%%%%%%%%%%%%%%%%%%%%%%%%%%%%%%%%%%%%%%%%%%%%%%%%%%%%% }
    
    %%%%%%%%%%%%%%%%% DO NOT CHANGE HERE %%%%%%%%%%%%%%%%%%%% {
    \pagestyle{fancyplain}
    \headheight 35pt
    \lhead{\NetIDa}
    \lhead{\NetIDa\\\NetIDb}                 
    \chead{\textbf{\Large Assignment \hwnumber}}
    \rhead{\course \\ \semester}
    \lfoot{}
    \cfoot{}
    \rfoot{\small\thepage}
    \headsep 1.5em
    %%%%%%%%%%%%%%%%%%%%%%%%%%%%%%%%%%%%%%%%%%%%%%%%%%%%%%%%%% }
    
    \begin{document}
     
    \section*{1.Computation Equation}
    % If the Problem is divided into items, use "enumerate"
    % Class-conditional:
    % \begin{equation}
    %     P_{x|\mu,\Sigma}=G(x,\mu,\Sigma) \nonumber
    % \end{equation}
    % where
    % \begin{equation}
    %     \Sigma=\frac{1}{N}\sum_{i=1}^{N}\left(x_i-\frac{1}{N}\sum_{i=1}^{N}
    %     x_i\right)\left(x_i-\frac{1}{N}\sum_{i=1}^{N}x_i\right)^T \nonumber
    % \end{equation}
    % Gaussian prior:
    % \begin{equation}
    %     P_{\mu}(\mu)=G(\mu,\mu_0,\Sigma_0) \nonumber
    % \end{equation}
    % where $\mu_0$ and $\Sigma_0$ is known. The following is the computation equations
    % of different classification methods.
    % \begin{enumerate}[a)]
    %     \item 
    %     predictive distribution:
    %     \begin{equation}
    %         P_{x|T}(x|D)=G(x,\mu_n,\Sigma+\Sigma_n) \nonumber
    %     \end{equation}
    %     where
    %     \begin{equation}
    %         \mu_n=\Sigma_0\left(\Sigma_0+\frac{1}{N}\Sigma\right)^{-1}
    %         \mu_{ML}+\frac{1}{N}\Sigma\left(\Sigma_0+\frac{1}{N}\Sigma\right)^{-1}
    %         \mu_0 \nonumber
    %     \end{equation}
    %     \begin{equation}
    %         \Sigma_n=\Sigma_0\left(\Sigma_0+\frac{1}{N}\Sigma\right)^{-1}
    %         \frac{1}{N}\Sigma \nonumber
    %     \end{equation}
    %     \item 
    %     MAP:
    %     \begin{equation}
    %         P_{x|T}(x|D)=G(x,\mu_n,\Sigma) \nonumber
    %     \end{equation}
    %     \item 
    %     ML:
    %     \begin{equation}
    %         P_{x|T}(x|D)=G(x,\mu_{ML},\Sigma) \nonumber
    %     \end{equation}
    % \end{enumerate}

    \section*{2.Curves of Classification Error vs $\alpha$}
    % If the Problem is divided into items, use "enumerate"
    % \begin{figure}[H]
    %     \centering 
    %     \subfigure{
    %     \label{Fig.D1andS1}
    %     \includegraphics[width=0.45\textwidth]{images/D1andS1.jpg}}
    %     \subfigure{
    %     \label{Fig.D1andS2}
    %     \includegraphics[width=0.45\textwidth]{images/D1andS2.jpg}}
    % \end{figure}

    \section*{3.Observation and Explaination}
    \begin{enumerate}[a)]
        \item 
        
    \end{enumerate}

    \section*{Appendix}
    The following is the Matlab code.

    \subsection{HW5\_solution.m}
    \lstset{style=mystyle}
    \lstinputlisting[language=Matlab]{HW5_solution.m}

    \subsection{HW5\_plot.m}
    \lstset{style=mystyle}
    \lstinputlisting[language=Matlab]{HW5_plot.m}

    \subsection{fun\_getinit.m}
    This function performs parameters initialization.
    \lstset{style=mystyle}
    \lstinputlisting[language=Octave]{myfunction/fun_getinit.m}

    \subsection{fun\_BDR.m}
    This function performs BDR classification using parameters we learn from EM.
    \lstset{style=mystyle}
    \lstinputlisting[language=Octave]{myfunction/fun_BDR.m}

    \subsection{fun\_EM.m}
    This function performs EM.
    \lstset{style=mystyle}
    \lstinputlisting[language=Octave]{myfunction/fun_EM.m}

    \subsection{fun\_error.m}
    This function performs error computation.
    \lstset{style=mystyle}
    \lstinputlisting[language=Octave]{myfunction/fun_error.m}

    \end{document}