%%%%%%%%%%%%%%%%% DO NOT CHANGE HERE %%%%%%%%%%%%%%%%%%%% {
\documentclass[12pt,letterpaper]{article}
\usepackage{fullpage}
\usepackage[top=2cm, bottom=4.5cm, left=2.5cm, right=2.5cm]{geometry}
\usepackage{amsmath,amsthm,amsfonts,amssymb,amscd}
\usepackage{lastpage}
\usepackage{enumerate}
\usepackage{fancyhdr}
\usepackage{mathrsfs}
\usepackage{xcolor}
\usepackage{graphicx}
\usepackage{listings}
\usepackage{hyperref}
\usepackage{float} 
\usepackage{subfigure}
\definecolor{codegreen}{rgb}{0,0.6,0}
\definecolor{codegray}{rgb}{0.5,0.5,0.5}
\definecolor{codepurple}{rgb}{0.58,0,0.82}
\definecolor{backcolour}{rgb}{0.95,0.95,0.92}

\lstdefinestyle{mystyle}{
    backgroundcolor=\color{backcolour},   
    commentstyle=\color{codegreen},
    keywordstyle=\color{magenta},
    numberstyle=\tiny\color{codegray},
    stringstyle=\color{codepurple},
    basicstyle=\ttfamily\footnotesize,
    breakatwhitespace=false,         
    breaklines=true,                 
    captionpos=b,                    
    keepspaces=true,                 
    numbers=left,                    
    numbersep=5pt,                  
    showspaces=false,                
    showstringspaces=false,
    showtabs=false,                  
    tabsize=2
}

\hypersetup{%
    colorlinks=true,
    linkcolor=blue,
    linkbordercolor={0 0 1}
}

\setlength{\parindent}{0.0in}
\setlength{\parskip}{0.05in}
%%%%%%%%%%%%%%%%%%%%%%%%%%%%%%%%%%%%%%%%%%%%%%%%%%%%%%%%%% }

%%%%%%%%%%%%%%%%%%%%%%%% CHANGE HERE %%%%%%%%%%%%%%%%%%%% {
\newcommand\course{ECE 271A}
\newcommand\semester{Fall 2019}
\newcommand\hwnumber{\#5}                 % <-- ASSIGNMENT #
\newcommand\NetIDa{Jiaming Lai}           % <-- YOUR NAME
\newcommand\NetIDb{A53314574}           % <-- STUDENT ID #
%%%%%%%%%%%%%%%%%%%%%%%%%%%%%%%%%%%%%%%%%%%%%%%%%%%%%%%%%% }

%%%%%%%%%%%%%%%%% DO NOT CHANGE HERE %%%%%%%%%%%%%%%%%%%% {
\pagestyle{fancyplain}
\headheight 35pt
\lhead{\NetIDa}
\lhead{\NetIDa\\\NetIDb}                 
\chead{\textbf{\Large Assignment \hwnumber}}
\rhead{\course \\ \semester}
\lfoot{}
\cfoot{}
\rfoot{\small\thepage}
\headsep 1.5em
%%%%%%%%%%%%%%%%%%%%%%%%%%%%%%%%%%%%%%%%%%%%%%%%%%%%%%%%%% }

\begin{document}
    
\section{1.Computation Equation}
The EM update equation we use is as following:
\begin{equation}
    h_{ij} = \frac{G(x_i,\mu_j^{(n)},\sigma_j^{(n)})\pi_j^{(n)}}
    {\sum_{k=1}^{C}G(x_i,\mu_k^{(n)},\sigma_k^{(n)})\pi_k^{(n)}}
    \nonumber
\end{equation}
\begin{equation}
    \mu_j^{(n+1)}=\frac{\sum_{i}h_{ij}x_i}{\sum_{i}h_{ij}}
    \nonumber
\end{equation}
\begin{equation}
    \pi_j^{(n+1)}=\frac{1}{n}\sum_{i}h_{ij}
    \nonumber
\end{equation}
\begin{equation}
    \sigma_j^{2(n+1)}=\frac{\sum_{i}h_{ij}(x_i-\mu_j)^2}{\sum_{i}h_{ij}}
    \nonumber
\end{equation}

\section{Problem (a)}
The following is the plot of PoE versus dimension for each of 25 classifiers obtained with all possible
mixture pairs. $FG$ represents the mixture of Gaussian to cheetah class and $BG$ represents the mixture
of Gaussian to grass class.
\begin{figure}[H]
    \centering 
    \subfigure{
    \label{Fig.FG1}
    \includegraphics[width=0.45\textwidth]{images/FG=1.jpg}}
    \subfigure{
    \label{Fig.FG2}
    \includegraphics[width=0.45\textwidth]{images/FG=2.jpg}}
\end{figure}
\begin{figure}[H]
    \centering 
    \subfigure{
    \label{Fig.FG3}
    \includegraphics[width=0.45\textwidth]{images/FG=3.jpg}}
    \subfigure{
    \label{Fig.FG4}
    \includegraphics[width=0.45\textwidth]{images/FG=4.jpg}}
\end{figure}
\begin{figure}[H]
    \centering 
    \label{Fig.FG5}
    \includegraphics[width=0.45\textwidth]{images/FG=5.jpg}
\end{figure}

\section{Problem (b)}
The following is the PoE versus dimension for each number of mixture components.
\begin{figure}[H]
    \centering 
    \label{Fig.MixtureC}
    \includegraphics[width=0.45\textwidth]{images/mix_C.jpg}
\end{figure}

\section*{Appendix}
The following is the Matlab code.

\subsection{HW5\_solution.m}
\lstset{style=mystyle}
\lstinputlisting[language=Matlab]{HW5_solution.m}

\subsection{HW5\_plot.m}
\lstset{style=mystyle}
\lstinputlisting[language=Matlab]{HW5_plot.m}

\subsection{fun\_getinit.m}
This function performs parameters initialization.
\lstset{style=mystyle}
\lstinputlisting[language=Octave]{myfunction/fun_getinit.m}

\subsection{fun\_BDR.m}
This function performs BDR classification using parameters we learn from EM.
\lstset{style=mystyle}
\lstinputlisting[language=Octave]{myfunction/fun_BDR.m}

\subsection{fun\_EM.m}
This function performs EM.
\lstset{style=mystyle}
\lstinputlisting[language=Octave]{myfunction/fun_EM.m}

\subsection{fun\_error.m}
This function performs error computation.
\lstset{style=mystyle}
\lstinputlisting[language=Octave]{myfunction/fun_error.m}

\end{document}